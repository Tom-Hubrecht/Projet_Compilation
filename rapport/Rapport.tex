\documentclass{article}

\usepackage[utf8]{inputenc}
\usepackage[T1]{fontenc}
\usepackage[french]{babel}
\usepackage[left=2cm, right=2cm, top=1cm, bottom=2cm]{geometry}
\usepackage{listings}
\usepackage{color}


%% Define colors for the listings %%
\definecolor{cred}{RGB}{255, 85, 85}
\definecolor{cgreen}{RGB}{85, 255, 85}
\definecolor{cblue}{RGB}{85, 85, 255}
\definecolor{cblack}{RGB}{0, 0, 0}
\definecolor{cwhite}{RGB}{255, 255, 255}


%% Listing Options %%
\lstdefinestyle{listing}{
	backgroundcolor=\color{cwhite},
	commentstyle=\color{cblack},
	keywordstyle=\color{cblack}\bfseries,
	numberstyle=\tiny\color{cblack},
	stringstyle=\color{cblack},
	basicstyle=\ttfamily\small,
	breakatwhitespace=false,
	breaklines=true,
	captionpos=b,
	keepspaces=true,
	numbers=left,
	numbersep=5pt,
	showspaces=false,
	showstringspaces=false,
	showtabs=false,
	tabsize=4
}

\lstset{style=listing}
\lstset{language=caml}

\author{Tom Hubrecht}
\date{}

\title{Rapport pour le projet de compilation}

\begin{document}

\maketitle

Tout ce qui a été demandé pour la première partie du projet a été compltété,
c'est à dire l'analyse lexicale, syntaxique ainsi que le typage d'un fichier
Petit Go passé en entrée.

\section{Choix techniques}

\subsection{Représentation d'un programme avec des AST}
J'ai choisi d'utiliser plusieurs types d'arbres de syntaxe abstraits (AST) pour
réaliser le projet. En effet, le parser renvoie un AST avec la localisation de
chaque noeud dans le fichier d'origine tandis que le typer convertit cet AST
localisé en AST typé, i.e. avec le type de chaque expression. De plus, le typer
transforme les appels à fmt.Print considérés par le parser comme des expressions
en instruction.

\subsection{Typage d'un programme}
Le typage du programme se déroule à peu près comme conseillé dans le sujet.
J'ajoute l'ensemble des structures à la liste des types valides qui sont au
départ uniquement \lstinline.[|int; bool; string|]. J'utilise ensuite cette
liste pour vérifier que les types présents dans le programme sont tous valides.
Les fonctions et les structures sont ajoutées dans des
\lstinline*Smap = Map.Make(String)* en vérifiant que tous leurs types sont
valides. Je vérifie qu'aucune structure récursive n'est présente en effectuant
un parcours en profondeur puis je vérifie qu'une fonction \lstinline*main* est
présente sans arguments ni valeur retournée. Enfin, je vérifie que toutes les
instructions du programme sont bien typées et ce faisant je construis une AST
typé pour faciliter la production de code future.

\section{Difficultées rencontrées}
Pour vérifier que chaque variable est bien utilisée, j'avais commencé par
stocker l'information dans une \lstinline*Smap* mais la déclaration de variables
dans un sous-bloc posait problème, j'ai donc choisi d'utiliser une
\lstinline*Hashtbl* où chaque variable est identifiée par son nom et sa
profondeur dans le programme.



\end{document}
